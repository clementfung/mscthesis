%!TEX root = main.tex
%%%%%%%%%%%%%%%%%%%%%%%%%%%%%%%%%%%%%%%%%%%%%%%%%%%%%%%%%%%%%%%%%%
\chapter{Introduction}
\label{sec:intro}
%%%%%%%%%%%%%%%%%%%%%%%%%%%%%%%%%%%%%%%%%%%%%%%%%%%%%%%%%%%%%%%%%%

\ac{ML} models rely on large volumes of diverse,
representative data for training and validation. However, to build
multi-party models from user-generated data, users must provide and
share their potentially sensitive information. Existing 
solutions~\cite{Low:2012, Li:2014}, even when incorporating
privacy-preserving mechanisms~\cite{Abadi:2016}, assume
the presence of a trusted central entity that all users share
their data and/or model updates with. For example, the Gboard
service~\cite{Gboard:2017} uses sensitive data from Android keyboard
inputs to generate better text suggestions; users who wish to train
an accurate Gboard suggestion model must send their mobile keyboard
data to Google.

Federated learning~\cite{McMahan:2017} keeps data on the client device
and trains ML models by only transferring model parameters to a trusted
central coordinator. In the quest for training the optimal ML model as
quickly as possible, the coordinator is not incentivized to provide
privacy for data providers: data is collected and is only kept on the
device as a performance optimization~\cite{McMahan:2017}.  

Furthermore, federated learning is not private or secure against
adversaries, who can pose as honest data providers or an honest
coordinator and who use auxiliary information from the learning process to
infer information about the training data of other 
users~\cite{Hitaj:2017}, despite data never being
explicitly shared. One may consider obfuscating data labels before
learning, but this is also insufficient to guarantee 
privacy~\cite{Calandrino:2011}. General privacy-preserving computation
models exist, but they rely on substantial amounts of additional
infrastructure. These include homomorphically encrypted secure
aggregation~\cite{Bonawitz:2017}, secure multi-party 
computation~\cite{Mohassel:2017}, or trusted \ac{SGX}~\cite{Ohrimenko:2016}, 
all of which are infeasible for
individuals and small organizations to deploy.

Today there is no accessible solution to collaborative multi-party
machine learning that maintains privacy and anonymity in an untrusted
setting. In developing this solution, we propose a novel setting
called \textit{brokered learning} that decouples the role of
coordination from the role of defining the learning process. We
introduce a short-lived, honest-but-curious \emph{broker} that mediates
interactions between a \emph{curator}, who defines the shared learning
task, and \emph{clients}, who provide training data. In decoupling
these roles, curators are no longer able to directly link clients to
their model updates, and cannot manipulate the learning
process.

Clients and curators never directly communicate: they are protected
from each other by a broker that is only used to coordinate the
learning task. The broker is administered by an honest-but-curious
neutral party, detects and rejects anomalous behavior, and terminates
when the learning objective is met. Our system design supports privacy
and anonymity by building on accessible public infrastructure to
minimize the cost of setting up and maintaining a broker.

We realize the brokered learning setting by designing, implementing,
and evaluating \emph{TorMentor}, a system which creates \emph{brokers}
to interface with the curator and the clients
in a brokered learning process. TorMentor is
implemented as a hidden Tor service, but can use any general-purpose
anonymous communication service to safeguard the identities of
curators and clients. % between themselves and from each other.

% TorMentor is functionally analogous to
% traditional distributed parameter servers in the federated learning 
% model~\cite{McMahan:2017}, but is unique in that it decouples the
% curator from the broker, supporting settings in which the curator is
% untrusted.

Although the model curator is removed from the learning process, a
myriad of other attacks are still possible. We adapt known ML attacks
to brokered learning and build on several state of the art techniques
to thwart a variety of these attacks when they are mounted by clients,
brokers and curators. Client-side differential
privacy~\cite{Dwork:2014, Geyer:2017} protects users from inversion
attacks~\cite{Fredrikson:2014, Fredrikson:2015}, 
\ac{RONI}~\cite{Barreno:2010} and monitored client
statistics~\cite{Mozaffari-Kermani:2015} prevent model poisoning
attacks~\cite{Biggio:2012, Huang:2011} and proof of
work~\cite{Back:2002} mitigates sybil attacks~\cite{Douceur:2002}.

Our evaluation of TorMentor demonstrates that these defenses protect
clients and curators from each other. For example, in one experiment
with 50\% malicious poisoning clients, a TorMentor broker was able to
converge to an optimum after mitigating and recovering from 
malicious behavior through our novel adaptive proof of work mechanism.
We also evaluated the performance of our prototype in a geo-distributed
setting: across 200 geo-distributed clients with 14 MB of data
per client, the training process in TorMentor takes 67s. By
comparison the training on a similar federated learning system without
Tor would take 13s. The observed overhead of TorMentor ranges from
5-10x, and depending on the level of privacy and security required,
TorMentor's modular design allows users to further tune the system to
meet their expected needs on the security-performance trade-off.

In summary, we make four contributions:
\begin{itemize}[label=$\star$]
 \item We develop a \textit{brokered learning} setting for
    privacy-preserving anonymous multi-party machine learning in an
    untrusted setting. We define the responsibilities, interactions,
    and threat models for the three actors in brokered learning:
    curators, clients, and the broker.

 \item We realize the brokered learning model in the design and
    implementation of \textit{TorMentor} and evaluate TorMentor's
    training and classification performance on a public dataset.

 \item We translate known attacks on centralized ML 
    (poisoning~\cite{Huang:2011, Nelson:2008} and 
    inversion~\cite{Fredrikson:2014, Fredrikson:2015}) and known
    defenses in centralized ML (RONI~\cite{Barreno:2010}, differential
    privacy~\cite{Dwork:2014}) to the brokered learning setting. We
    evaluate the privacy and utility implications of these attacks and
    defenses.

 \item We design, implement, and evaluate three new defense mechanisms
    for the brokered learning setting: distributed RONI,
    adaptive proof of work, and thresholding the number of clients.

\end{itemize}

