%%%%%%%%%%%%%%%%%%%%%%%%%%%%%%%%%%%%%%%%%%%%%%%%%%%%%%%%%%%%%%%%%%
\section{Related work}
\label{sec:related}
%%%%%%%%%%%%%%%%%%%%%%%%%%%%%%%%%%%%%%%%%%%%%%%%%%%%%%%%%%%%%%%%%%

\clement{This section will be gone. It includes content that I haven't
added to the background/decided to delete yet.}

\textbf{Prio.}
Prio (Private, Robust and Scalable Computation of Aggregate
Statistics) \cite{Corrigan-Gibbs:2017} is a system that makes weaker 
assumptions about the trust between clients and curators. Prio assumes
that clients or curators can be malicious, and builds effective
defenses against these cases. As long as one curator is honest, Prio is
able to compute aggregate values of the private values of several
clients. Prio has functionality that supports machine learning, but it
fundamentally relies on using private sums to train models, and
computing the solution to a large linear system. This is only effective
for settings in which the loss function can be represented as such:
when the function is convex and differentiable, but breaks down for
techniques that fail to hold this property. Gradient descent is an
alternative solution for solving systems in this setting and
generalizes to functions that cannot be solved as a system of linear
equations. In order to properly handle a wide variety of machine
learning workloads, the ability to perform gradient descent is an
essential component. 
%
\ivan{Wait, so the only diff. from Prio is that we do this for
  gradient descent? This seems a bit narrow. I hope there are other
  differences..}

\clement{I have to think of the best way to flow into and incorporate
this. It's a bit tough because Prio doesn't introduce any primitives
that we actually incorporate or build on in our system. The main thing
is that Prio has a similar trust model to ours, but has much less
utility. It only computes sums. }

\ivan{One strategy is to expand the intro with more general-purpose
  cites, such as Prio. Think of the middle portion of the intro as
  your related work section that provides general background. Then,
  the background section can get into the super detailed pieces that
  are essential to (understanding/contextualizing) this paper.}
